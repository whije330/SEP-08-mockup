\chapter{Usage Guide}
\label{chap:usage_guide}

The use of the software \emph{iFire} is to accelerate the procedure of alerting
encounters of fire. The software will also provide a better understanding of the
situation with the included functions of taking pictures and sending GPS
coordinates.

Currently the actors which perform said action are the victims or the witnesses.
They can report a fire encounter using \emph{iFire}. The application is designed
to help in every situation possible while also beeing very user friendly.

%This section is aimed at describing the general use of the software. Such
%information is grouped by the different kinds of actors.
%Such actors are expected to use the software to perform some
%processes or workflows (called here procedures) using the concerned software
%\textbf{(including installation procedures)}.

%The description of the processes should be organised to facilitate learning by
%presenting simpler, more common, or initial processes before more complex, less
%utilised, or subsequent processes.

%Common procedures should be presented once to avoid redundancy when they are
%used in more complex procedures. 

%Each process has to be documented using the following use-case textual
% description template \cite{armour01usecase} \textbf{BUT its content must be as low level as possible with actual values}:
%\vspace{0.5cm}
%\hrule
%\begin{lyxlist}{UC1}
%\small{
%\item [\textbf{Use~Case:}] ProcessMissionOne
%\item [\textbf{Scope:}] Crisis Management System (\emph{CMS})
%\item [\textbf{Primary Actor}:] Coordinator John
%\item [\textbf{Secondary Actor}:] FirstAidWorker Bob,\\
%                  ExternalResourceSystem (ERS)
%\item [\textbf{Intention:}]The intention of the Coordinator is to process
% mission with ID equal to 1.
%\item [\textbf{Level}:]Sub-functional level
%\item [\textbf{Main~Success~Scenario}]:\\
%1. \emph{John} instructs the \emph{CMS} to process a specific mission.\\
%2. \emph{CMS} selects the internal worker \emph{Bob} to execute the mission.\\
%3. \emph{CMS} instructs `\emph{Bob} to behave as \emph{FAW}.\\
%4. \emph{Bob} informs to the \emph{CMS} of his arrival.\\
%5. \emph{Bob} executes the mission.\\
%6. \emph{Bob} informs to the \emph{CMS} the mission outcome.


%\item [\textbf{Extensions}]:\\
%2.a None internal worker can execute the mission.\\
%\hspace*{0.5cm} 2.a.1 \emph{CMS} requests an external resource to \emph{ERS}.\\
%\hspace*{0.5cm} 2.a.2 \emph{ERS} informs \emph{CMS} that the request can be
% processed.\\
%\hspace*{1.4cm} Use case continues at step 3.

%}

%\end{lyxlist}
%\hrule
%\vspace{0.5cm}

%\Remark{Graphical User Interfaces (GUIs)}: include GUIs screenshots to show the
%different stages of the process while its is performed by the actor.



\section{Actors common procedures}
%Common procedures to several actors are grouped in this section to avoid
%redundancy.

Victim/Witness uses the \emph{Victim/Witness alert fire} to indicate the
encounter of a \emph{fire}. The GUI is almost exactly the same, the difference
lies in the requested data. For example the victim can sometimes say more about
the fire than the witness because he/she is trapped inside the building. This is
all done by working with the iFire application.
When the user finished inputting the requested data the smartphone will forward
it to the server.

The server will then evaluate the received data and sends parts of the data to
the respective departments, these departments are \emph{policemen},
\emph{firefighters} and \emph{medics}.

\subsection{Victim/Witness alerts fire}

\vspace{0.5cm}
\hrule
\begin{lyxlist}{UC1}
\small{
\item [\textbf{Use~Case:}] Victim/Witness alerts fire
\item [\textbf{Scope:}] Crisis Management System (\emph{CMS})
\item [\textbf{Primary Actor}:] Victim/Witness
\item [\textbf{Secondary Actor}:] Departments,\\
                  Victim/Witness smartphone (VSP),\\
                  Server
\item [\textbf{Intention:}]The intention of the Victim/Witness is to create a
mission.
\item [\textbf{Level}:]Sub-functional level
\item [\textbf{Main~Success~Scenario}]:\\
1. \emph{Victim/Witness} fills out the \emph{CMS} to process a mission with the help
of \emph{VSP}.\\
2. \emph{VSP} sends the data to the \emph{Server} to inform of a new
mission.\\
3. \emph{Server} sends the respective data to the respective
\emph{Departments}.\\
4. \emph{Departments} executes the mission.\\
5. \emph{Departments} informs to the \emph{Server} of his arrival.\\
6. \emph{Departments} informs to the \emph{Server} the mission outcome.


\item [\textbf{Extensions}]:\\
4.a None internal worker can execute the mission.\\
\hspace*{0.5cm} 4.a.1 \emph{Departments} informs \emph{Server} that the request
can be processed.\\
\hspace*{1.4cm} Use case continues at step 5.

}

\end{lyxlist}
\hrule
\vspace{0.5cm}

\subsection{MyCommonProcedure2}

Not yet used.

\section{Firefighter reports}

The actor \emph{firefighter} reports to the server if he is able to execute
the mission. If he is capable of doing so, the firefighter needs to report how
many men will be sent to the mission. These men will all be categorized as
\emph{firefighters} on our server.

\subsection{Firefighter reporting}

\vspace{0.5cm}
\hrule
\begin{lyxlist}{UC2}
\small{
\item [\textbf{Use~Case:}] Firefighter reporting
\item [\textbf{Scope:}] Crisis Management System (\emph{CMS})
\item [\textbf{Primary Actor}:] Firefighter
\item [\textbf{Secondary Actor}:] Server
\item [\textbf{Intention:}]The intention of the Firefighter is to confirm and
process a mission.
\item [\textbf{Level}:]Sub-functional level
\item [\textbf{Main~Success~Scenario}]:\\
1. \emph{Firefighter} confirms the new listed mission.\\
2. \emph{Firefighter} sends his men to the missions' location and informs the
\emph{Server} how many he sent.\\
3. \emph{Server} groups the received data to the respective tables.\\
4. \emph{Firefighter} informs to the \emph{Server} of his arrival.\\
5. \emph{Firefighter} reports to the \emph{Server} the mission outcome and
details.

}

\end{lyxlist}
\hrule
\vspace{0.5cm}

\subsection{MyProcedure2}




\section{My-Actor2 procedures}
\subsection{MyProcedure1}
\subsection{MyProcedure2}


\section{My-Actor3 procedures}

\subsection{MyProcedure1}
\subsection{MyProcedure2}