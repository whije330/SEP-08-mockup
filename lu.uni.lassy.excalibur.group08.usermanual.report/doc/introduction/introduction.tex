\chapter{Introduction}
\label{chap:introduction}

\section{Scope}

This document provides basic information about \mysystemname.The document
contains environments where \mysystemname can be deployed, information how
victims and witnesses of a fire can request help with \mysystemname.

This document may be used with other documents provided by third-party companies
which provide a better understand in which cases and environment
where the software \mysystemname is supposed to be deployed.

This document is not intended to provide information on how to download,
install, or configure \mysystemname.


%This document provides \ldots
%Example: This document provides minimum acceptable information for knowing how
% to use the software system \mysystemname.


%This document does not \ldots 
 
%This document is not \ldots
%Example: This document is not intended to provide information about how to
% connect, deploy, configure, or use any external device or
% third-party software system that is rqeuired for the correct funcitoning of
% \mysystemname.

 
%This document may be used with \ldots
%This document may be used with other documents provided by third-party
% companies to have an overall view and correct understanding of the environment
% and procedures where the software system \mysystemname is aimed to be deployed
% and run.




\section{Purpose}
We aim for a quick and accurate transmission of information to the central
stations. Our project is user friendly and helps to minimise panic situations.
There is no more need to call the central station, with only a few clicks you can provide
all the useful information the stations need for an efficient deployment of all
rescue workers.

\section{Intended audience}
The intended audience for our application is evrey person that could be a victim
or witness of a fire. The application itself is designated for everybody and the
tracking software for firefighters in their central station.  

\section{\mysystemname}
MyFire is an application for smartphones which indicates and locates the
situation where fire is burning by taking a pictures and automatically sending
the GPS coordinates to the central stations.


\subsection{Actors \& Functionalities}
Our application has a simple user-friendly interface which can be use by anybody
with just a few clicks. Of course you can choose to call the central station if
needed. For our software which is located at the fire departments are also
intended for end-users with a purposeful interface where an interaction with the
witness is possible.


\subsection{Operating environment}
The software is deployed in fire departments and will be used by firefighters to
locate and categorize the situation which will be send by a witness or victim
via smartphone application.

\section{Document structure}  
This user manual is guideline for firefighters and others who use our smartphone
application. In chapter 3 there will be application instructions as well as
procedures on how the application is working. Chapter 4 describes our software
more in depth.




