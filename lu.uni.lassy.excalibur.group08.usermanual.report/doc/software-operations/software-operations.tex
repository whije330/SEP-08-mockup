\chapter{Software operations}
\label{chap:soptware_operations}


Our project has three parts: smartphone application, tracking software and our
servers. The application can be used by any person and the software will be used
by a person in authority at the fire department. The tracking software will be
triggered by a notification as an alert as soon as the smartphone application is
being used. It will indicate the location with the picture as well as the date.
The person in authority will decide which security level will be taken as well
as further needed procedures. After receiving the picture, the person in
authority will need to call the person who took the picture to obtain more
details on the situation or to conciliate the person. The gathered data will be
saved on our servers as a backup or for case investigations but never for public
media purposes. Notice: The taken pictures will never be saved on your
smartphone for privacy reasons.


\section{MyOperation}
\label{operation:MyOperation}
The system operator creates and adds a new fire to the system after being
informed by a third party (citizen, witness) and selects diffrent rescue workers
for the fire.

\begin{description}

\item \textbf{Parameters:} Witness or victim, Information about
fire, Rescue workers
\item \textbf{Precondition:} The system operator is logged in and has received information from a reporter.
\item \textbf{Post-condition:} A new fire has been added to the system and the
new fire has been assigned to rescue worker, the rescue worker has received an
automatic notification from the system.
\item \textbf{Output messages:} The selected rescue worker will be notified
automatically once the crisis has been created.

\item \textbf{Triggering:}
\begin{enumerate}
\item From within the crisis management window fill out the required entries
related to the personal information of the vicitm or witness.
\item Fill out the entries related to the fire and if you are a victim or a
witness. If you are a victim you need to report if you are alone, trapped and
send a photo. If you are witness you need to report what is bruning. Both
witness and victim need to report what is burning and what the cause of the
fire is.
\item Click on the Send button and the fire will be reported to the central
station.
\end{enumerate}

 
\end{description}

 
\subsection{Apllication usage}
Starting with an example of smartphone application:
\\[8pt]
A walking person who sees a burning barn of a farmer can quickly use his
application without waiting for an answer on a call or describing the location
of the incident. At the start he will need to define himself as a witness and
indicate what is burning. After, he will need to confirm that he is sure to
send the data, then he needs to take a picture which will be send to the next
fire department. Additional details will be shown to help specify the situation.
The person will receive immediately a call from the department to communicate
with him as the witness for extra details or to settle down the person in panic
situations if needed.

\subsection{Tracking software}
Example of our tracking software:
\\[8pt]
The person in authority receives an alert at the fire department. A notification
pops up on the screen with a map location, the time and the picture from the
crash scene. That person justifies the right emergency level and passes the
given details to the firefighters. To clarify the situation, he or she calls
that person for further information or just to calm down the witness.



